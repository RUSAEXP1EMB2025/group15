\documentclass{jarticle}
\usepackage[dvipdfmx]{graphicx} % Required for inserting images
\usepackage{amsmath}
\usepackage{tikz}
\usepackage{float}
\usepackage{caption}
\usepackage{capt-of}
\usepackage{amsmath}
\usepackage{listings}


\title{要求仕様書}
\date{}

\begin{document}
\maketitle

\section*{全体概要}
\subsection*{システムの概要}
LINE から設定できるスマートスピーカー, 照明による目覚まし機能, 及び部屋の明るさの自動調整.
\subsection*{製品の機能}
目覚まし機能はスマートスピーカーと照明により行われる.LINE から鳴らしたい音, スヌーズ機能, 曜日ご
との設定, 日ごとの設定, アラームの解除ができる. 照明はその日の天気が分かるよう晴れなら冷色, 雨なら暖
色, 曇りならその中間の色の明かりがつく. 部屋の明るさの自動調整は, 5分ごとに Remo3 のセンサから光
度, 人がいるかの情報を取得し, 人がいた場合には光度を一定に保つ. 夜に電気が消えた場合は調整が行われな
い.LINE から明るさ, 調整の有無を設定が出来る.
\subsection*{想定する利用者の特性}
明かりの要素がある目覚まし機能であり, その調整をスマホで遠隔で行えるため, 目覚まし機能を求める人を
想定している. また部屋の明るさの自動調整は, 日中は日の光で生活し, 暗くなってきた時の調整が面倒な人に
とって便利だと考える.

\section*{詳細}
\subsection*{機能要求}
\begin{itemize}
    \item ユーザは LINE から目覚ましの機能 (鳴らしたい音, スヌーズ機能, 曜日ごとの設定, 日ごとの設定, アラームの解除) が設定できること.\\
    \item 目覚ましの明かりの色によってその日の天気が分かること.\\
    \item 部屋に人がいる時に明るさの自動調整が行われること.\\
    \item ユーザはLINE から照明の設定(明るさ, 調整の有無) ができること.\\
\end{itemize}
\end{document}