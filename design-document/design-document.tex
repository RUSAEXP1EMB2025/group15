\documentclass{jarticle}
\usepackage[dvipdfmx]{graphicx} % Required for inserting images
\usepackage{amsmath}
\usepackage{tikz}
\usepackage{float}
\usepackage{caption}
\usepackage{capt-of}
\usepackage{amsmath}
\usepackage{listings}


\title{設計書}

\begin{document}
\maketitle

\section*{設計内容の概要}
\begin{itemize}
    \item ユーザーは、LINEメッセージで目覚ましアラームを「MMDDhhmm」形式で指定する。スヌーズ機能や曜日指定にも対応し、アラームの有効・無効を切り替えることも可能。\\
    \item Remo 3 のセンサと外部API(気象庁・天文台API)を利用し、以下の情報を毎日取得:\\
        \begin{enumerate}
            \item 日の出・日没時刻\\
            \item 天気(晴れ・曇り・雨)\\
            \item 室内の照度(Remo 3センサより)\\
        \end{enumerate}
    \item 取得情報に基づいて以下の制御を行う:\\
        \begin{enumerate}
            \item 日の出時刻:目覚まし時刻の設定がない場合照明ON\\
            \item 日没1時間前:ユーザーが在宅の場合のみ照明ON\\
            \item 天気による日の出及び起床時に点灯する照明色変化(晴れ=白、曇り=青白、雨=黄)\\
            \item 室内の輝度に応じて照明の明るさ自動調節\\
        \end{enumerate}
\end{itemize}

\section*{システム処理の流れ}
\begin{itemize}
    \item 日次初期化処理(午前0時に実行)\\
        \begin{enumerate}
            \item 日次初期化処理(午前0時に実行)\\
            \item 天気APIから天気を取得 \\
            \item スケジュール登録(Google Calendar または内部配列)\\
        \end{enumerate}
    \item センサ定期チェック(5分おき)\\
        \begin{enumerate}
            \item Remo 3から照度取得\\
            \item 明るさに応じて照明ON/OFFまたは調光を実行\\
        \end{enumerate}
    \item 目覚まし制御 \\
        \begin{enumerate}
            \item LINEから指定された時刻を受信・解析\\
            \item 指定時刻にアラーム動作(照明ONまたは音源操作)\\
            \item スヌーズ(例:5分後再実行)も可能 \\
            \item 曜日指定時はその曜日にのみ実行 \\
        \end{enumerate}
    \item  天気と照明色の連動\\
        \begin{enumerate}
            \item 天気APIから情報取得\\
            \item 照明の色をプリセットに応じて変更(Remo登録済み赤外線信号で実現)\\
        \end{enumerate}
    \item 日没時の照明点灯\\
        \begin{enumerate}
            \item 日没1時間前に人感センサーまたは在宅フラグを確認\\
            \item 在宅なら照明点灯、いなければスキップ\\
        \end{enumerate}
\end{itemize}

\section*{必要なモジュール(.gsファイル)}

\begin{enumerate}
  \item \textbf{line\_handler.gs}
  \begin{itemize}
    \item LINE Webhook の受信
    \item ユーザーからのコマンド(MMDDhhmm形式など)の解析
    \item スケジュールデータの登録・管理
  \end{itemize}

  \item \textbf{schedule\_manager.gs}
  \begin{itemize}
    \item 目覚ましスケジュールの登録と実行
    \item 日の出・日没イベントの管理
    \item 曜日指定やスヌーズ対応など
  \end{itemize}

  \item \textbf{sensor\_reader.gs}
  \begin{itemize}
    \item Nature Remo 3 からの照度センサーデータ取得
    \item 室内明るさに応じた照明レベルの判定
  \end{itemize}

  \item \textbf{weather\_fetcher.gs}
  \begin{itemize}
    \item 気象庁APIやOpenWeatherMapからの天気情報取得
    \item 天気データの解析(晴れ・曇り・雨の判定)
  \end{itemize}

  \item \textbf{alarm\_controller.gs}
  \begin{itemize}
    \item アラーム時の照明ONや赤外線信号の送信
    \item スヌーズ機能の実装
  \end{itemize}

  \item \textbf{light\_controller.gs}
  \begin{itemize}
    \item 天気や照度、日の出・日没時間に応じた照明制御
    \item 照明の色や明るさの変更処理
  \end{itemize}
\end{enumerate}


\end{document}